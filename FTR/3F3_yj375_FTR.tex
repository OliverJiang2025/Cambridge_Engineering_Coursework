\documentclass[12pt]{article}[times]

\topmargin 0.0cm 
\oddsidemargin 0.0in 
\evensidemargin0.0in
\textheight 22cm 
\textwidth  17cm 
\headheight 0in 
\headsep 0in
\parindent0in

\usepackage{amsmath}
\usepackage{amsfonts}
\usepackage{amssymb}
\usepackage{graphicx}
\usepackage{tikz}
\usepackage{amssymb}
\usepackage{circuitikz}
\usepackage{setspace}
\doublespacing{}
\begin{document}

%\hspace{1cm}

\begin{center}
\Large{\bf CAMBRIDGE UNIVERSITY ENGINEERING DEPARTMENT}
\end{center}

\vskip 1cm

\begin{center}
\large{\bf Part IIA Full Technical Report}
\end{center}




\vskip 1cm
\begin{center}
\fbox{\rule[0.0cm]{0cm}{1.0cm}
 \large{\bf 3F3 Random Number Generation}\rule[-0.75cm]{0cm}{1.0cm} }
\end{center}


\vskip 1cm

\begin{center}

Name: Yongqing Jiang \\
\vskip 0.2cm
CRSid: yj375
\vskip 0.2cm
 College: Peterhouse \\
  \vskip 0.2cm

%
Date of Experiment: Nov. 2025
%
\end{center}

\vskip 1cm

\section{Introduction}
This lab activity investigates
statistical methods by generating 
random numbers with underlying 
distribution. Uniform and normal
random variables are generated,
and they are visualized by histogram
and kernel smoothing density (KSD) function.
Functions of random variables 
are also discussed.
Inverse CDF method is used to 
generate random variables 
from any arbitrary distributions.
Finally, $\alpha$-stable 
distribution is discussed. 

\section{Methods, Results, and Discussion}
\subsection{Uniform and normal random variables}
In this section, uniform and normal random variables
are generated and visualized by using histogram
and kernel density function. The results are
compared and discussed.
\subsubsection{Histogram and kernel density function}
\begin{figure}[htbp]
    \centering
    \includegraphics[width=1\textwidth]{Figures/Figure_1.png}
    \caption{Histogram and KSD of uniform and normal random variables}
    \label{fig:fig1}
\end{figure}

In Figure \ref{fig:fig1}, 1000 random Gaussian numbers
and 1000 random uniform numbers are generated.
The histograms are generated with 50 bins and
the KSD functions has a width of 0.4 with a 
Gaussian kernel, in the form of\cite{labsheet}:
\begin{equation*}
  \pi_{KS}(x) = \frac{1}{N} \sum_{i=1}^{N} \frac{1}{\sigma} \mathcal{K}\left( \frac{x-x^{(i)}}{\sigma} \right)
\end{equation*}

Where $\mathcal{K}(\cdot) \sim \mathcal{N}(\cdot | 0,1)$, $\sigma$ is the width of the kernel.

For Gaussian random variables, the KSD provides a smooth approximation,
which closely follows the shape of the theoretical Gaussian distribution curve.
The histogram, on the other hand, shows more details and fluctuations 
due to the discrete nature of the bins. 

At the same time, the histogram
may not align with the theoretical curve perfectly due to the 
choice of bin width and the number of samples.

On the other hand, the uniform random variables show a different behavior
in histogram and KSD. The shape of the histogram is more likely to be
affected by the bin width, leading to a step-like appearance.
Also, it is prone to be affected by random fluctuations.

The KSD, however, smooths out these fluctuations and provides a more
continuous representation of the uniform distribution. At the same time,
it can lead to deviation from the ideal uniform shape,
due to the discontinuities at the edges of the uniform distribution.
This is because the kernel width is comparable to the range of the uniform distribution.

To discuss more about the effect of kernel width, several KSDs with distinct kernel widths
are plotted. 

\begin{figure}[htbp]
  \centering
  \includegraphics[width=1\textwidth]{Figures/Figure_2.png}
  \caption{KSD of uniform random variables with different kernel widths}
  \label{fig:fig2}
\end{figure}

From Figure \ref{fig:fig2}, we can see that with a small kernel width (0.05 and 0.1),
the KSD captures more details of the uniform distribution. It is reasonable
to state that this is a uniform distribution only by looking at the KSD.
However, with a larger kernel width that is comparable to the range of the uniform distribution
, the KSD smooths out the details and deviates from the ideal uniform shape.
To conclude, there is a trade-off when choosing the kernel width between 
smoothing and shape deviation. 


\subsubsection{Multinomial distribution}

The Multinomial distribution is a generalization of the binomial distribution \cite{Multinomial},
and it can be used to discribe a distribution within a histogram. 

Suppose there are some fixed finite number of bins, $J$, and
there are $N$ samples to be placed into these bins. 
Each bin has a probability $p_j$ of receiving a sample,
where $j = 1, 2, \ldots, J$ and $\sum_{j=1}^{J} p_j = 1$.

And $p_j$ can be calculated by integrating the
underlying probability density function $p(x)$.
\begin{equation*}
  p_j = \int^{c_j+\delta /2}_{c_j-\delta /2} p(x) dx
\end{equation*}
Where $c_j$ is the center of bin $j$, and $\delta$ is the width of each bin.

Let random variable $\vec{X} = (X_1, X_2, \ldots, X_J)$
represents the number of samples in each bin.

Then the PMF of the Multinomial distribution is given by:
\begin{equation*}
  P(\vec{X} = \vec{n}) = \frac{N!}{n_1! n_2! \ldots n_J!} p_1^{n_1} p_2^{n_2} \ldots p_J^{n_J}
\end{equation*}
Where $\vec{n} = (n_1, n_2, \ldots, n_J)$. When $\dim{\vec{X}}$ is 2, 
it reduced to a binomial distribution. 

The term $p_j^{n_j}$ represents the probability of $n_j$ samples falling into bin $j$,
as each sample has a probability $p_j$ of falling into that bin.

The factorial terms account for the different arrangements of samples across the bins,
considering that the order of samples within each bin does not matter.

Expectation of $X_j$ is given by:
\begin{equation*}
  \mathbb{E}[X_j] = N p_j
\end{equation*}
And Variance of $X_j$ is given by:
\begin{equation*}
  \text{Var}(X_j) = N p_j (1 - p_j)
\end{equation*}

For uniform distributed random variables, $p(x) = \frac{1}{N}$,
so $p_j = \frac{\delta}{N}$. 

$\mathbb{E}(X_j) = \frac{N \delta}{N} = \delta$,
$\text{Var}{(X_j)} = N \frac{\delta}{N} (1 - \frac{\delta}{N}) = \delta (1 - \frac{\delta}{N})$.

Three sets of uniform random variables are generated with N = 100, 1000, 10000 respectively.
The histograms are plotted with 30 bins, and the results are shown in Figure \ref{fig:fig3}.

\begin{figure}[htbp]
  \centering
  \includegraphics[width=1\textwidth]{Figures/Figure_3.png}
  \caption{Histograms of uniform random variables with different sample sizes}
  \label{fig:fig3}
\end{figure}

We can see that the normalized mean of each bin (height of each bar) is fixed
by the bin width, which is $\frac{1}{30} \approx 0.0333$.

For N = 1000 and 10000, the histogram bars fluctuate around this value
and fits in the interval $[\mu - 3\sigma, \mu + 3\sigma]$, as expected.

However, for N = 100, the histogram bars show significant deviation from the expected range.
This is because with a small sample size, the variance is relatively large compared to the mean.
It is implied that a small sample size may not accurately represent the underlying distribution
by using a histogram.

This analysis can also be applied to Gaussian random variables.

To start with,
\begin{equation*}
  p_j = \int^{c_j+\delta /2}_{c_j-\delta /2} p(x) dx = F(c_j+\delta /2) - F(c_j-\delta /2)
\end{equation*}
Where $F(x)$ is the CDF of the Gaussian distribution. This expression helps to 
evaluate $p_j$ in \textit{python} using \textit{scipy.stats.norm.cdf} function.

Then, $\mathbb{E}(X_j)$ and $\text{Var}(X_j)$ can be calculated accordingly.

\begin{figure}[htbp]
  \centering
  \includegraphics[width=1\textwidth]{Figures/Figure_4.png}
  \caption{Histograms of Gaussian random variables with different sample sizes}
  \label{fig:fig4}
\end{figure}

As shown in Figure \ref{fig:fig4}, the histograms of Gaussian random variables
are plotted with corresponding expected mean and variance for each bin height. 
Similar to the uniform distribution case, it is expected that the bar heights
lie within the interval $[\mu - 3\sigma, \mu + 3\sigma]$.

However, in this case, it can be observed that $\sigma$ is no longer a constant
throughout the range of random variables, since $p_j$ varies for different bins.
$\text{Var}(X_j) = Np_j(1-p_j)$. The variance is larger at the center of the distribution
where $p_j$ is larger, and smaller at the tails where $p_j$ approaches to 0 and 1.

\begin{figure}[htbp]
  \centering
  \includegraphics[width=1\textwidth]{Figures/Figure_5.png}
  \caption{Comparing variance of histogram height and $p_j$}
  \label{fig:fig5}
\end{figure}

This argument is visualized in Figure \ref{fig:fig5},
where the variance of histogram height and $p_j$ are plotted together.
It is clear that the variance is small when $p_j$ is small and at the tails. 

Due to the limited bin size, $\max{(p_j)}$ is around 0.1.
In this range, $\text{Var}(X_j) \approx N p_j$, which agrees 
with the linear trend shown.

\section{Function of Random Variables}


The Jacobian formula for change of variables in probability density functions 
states that if $y = f(x)$ is a differentiable and invertible function,
then the probability density function of the transformed random variable $y$ is given by:
\begin{equation*}
    p(y) = \frac{p(x)}{|dy/dx|}\bigg|_{x = f^{-1}(y)}
\end{equation*}

For a linear transformation,
$y = f(x) = ax + b \Rightarrow x = f^{-1}(y) = \frac{y-b}{a}$, and 
$dy/dx = a$.
With $f(\cdot)$ being the standard Gaussian distribution:
\begin{equation*}
    p(x) = \frac{1}{\sqrt{2\pi}}e^{-\frac{x^2}{2}}
\end{equation*}

\begin{equation*}
    p(y) = \frac{1}{|a|} \cdot \frac{1}{\sqrt{2\pi}} e^{-\frac{(\frac{y-b}{a})^2}{2}} = \frac{1}{\sqrt{2\pi a^2}} e^{-\frac{(y-b)^2}{2a^2}}
\end{equation*}



For a general normal distribution ${\cal N}(x|\mu,\sigma^2)$,
the probability density function is:
\begin{equation*}
    p(x) = \frac{1}{\sqrt{2\pi \sigma^2}} e^{-\frac{(x-\mu)^2}{2\sigma^2}}
\end{equation*}
We can see that the transformed random variable is ${\cal N}(y|b,a^2)$,
the linear transformation of a normal distribution is still a normal distribution,
with a shift of mean from $\mu$ to $\mu + b$ and a scaling of variance from $\sigma^2$ to $a^2 \sigma^2$.

\begin{figure}[htbp] 
    \centering  
    \includegraphics[width=1\textwidth]{Figures/Figure_6.png}   
    \caption{Histogram of linearly transformed Gaussian random variables and corresponding PDF}
    \label{fig:fig6}
\end{figure}

We can see from Figure \ref{fig:fig6} that the histogram of transformed random samples with
a = 6 and b = 3 fits well with the probability density function with
variance 6 and mean 3.


Now, consider a non-linear transformation, $y = f(x) = x^2$.
\begin{equation*}
  x = f^{-1}(y) = \begin{cases}
    \sqrt{y}, & x \geq 0 \\ 
    -\sqrt{y}, & x \leq 0
  \end{cases}
\end{equation*}
We can see that $f(\cdot)$ is not one-to-one,
so we need to consider both branches of the inverse function.
Also, $|dy/dx| = 2|x| = 2 \sqrt{y}$.
\begin{align*}
    p(y) &= \frac{p(x)}{|dy/dx|}\bigg|_{x = f^{-1}(y)} + \frac{p(x)}{|dy/dx|}\bigg|_{x = -f^{-1}(y)} \\
    &= \frac{1}{2\sqrt{y}} \cdot \frac{1}{\sqrt{2\pi}} e^{-\frac{(\sqrt{y})^2}{2}} + \frac{1}{2\sqrt{y}} \cdot \frac{1}{\sqrt{2\pi}} e^{-\frac{(-\sqrt{y})^2}{2}} \\
    &= \frac{1}{\sqrt{2\pi y}} e^{-\frac{y}{2}}
\end{align*}

\begin{figure}[htbp] 
    \centering  
    \includegraphics[width=0.7\textwidth]{Figures/Figure_7.png}
    \caption{Histogram of squared Gaussian random variables and corresponding PDF}
    \label{fig:fig7}
\end{figure}

As shown by Figure \ref{fig:fig7}, the theory agrees well to the 
histogram of squared Gaussian random variables.

Lastly, consider $p(x) = \mathcal{U}(x|0,2\pi)$,
and transformation $y = f(x) = \sin(x)$.
The inverse of $f(\cdot)$ is also not one-to-one within the range of $x$.

There are four branches of the inverse function within $[0,2\pi]$,
and it is better to defind $\text{Arcsin}(\cdot)$ function 
that maps $[0, 1] \to [0, \frac{\pi}{2}]$.
\begin{equation*}
  x = f^{-1}(y) = \begin{cases}
    \text{Arcsin}(y), & 0 \leq x \leq \frac{\pi}{2}, 0 \leq y \leq 1 \\ 
    \pi - \text{Arcsin}{(y)}, & \frac{\pi}{2} \leq x \leq \pi, 0 \leq y \leq 1 \\
    \pi + \text{Arcsin}(-y), & \pi \leq x \leq \frac{3\pi}{2}, -1 \leq y \leq 0 \\
    2\pi - \text{Arcsin}(-y), & \frac{3\pi}{2} \leq x \leq 2\pi, -1 \leq y \leq 0
  \end{cases}
\end{equation*}

Also, $|dy/dx| = |\cos(x)|$.

Now, the transformed PDF can be calculated as:
\begin{equation*}
  p(y) = \sum^{2}_{i=1} \frac{p(x)}{|dy/dx|}\bigg|_{x = f^{-1}_i(y)} = \sum^{2}_{i=1} \frac{1}{2\pi |\cos(x)|}\bigg|_{x = f^{-1}_i(y)}, \quad 0 \leq y \leq 1
\end{equation*}

$|\cos(x)| = \sqrt{1-y^2}$ for all four branches of $f^{-1}(\cdot)$, so 
\begin{equation*}
  p(y) = \frac{2}{2\pi \sqrt{1-y^2}} = \frac{1}{\pi \sqrt{1-y^2}}, \quad 0 \leq y \leq 1
\end{equation*}

Similarly, $p(y) = \frac{1}{\pi \sqrt{1-y^2}}, \quad -1 \leq y \leq 0$.
Combining both parts, we have:
\begin{equation*}
  p(y) = \frac{1}{\pi \sqrt{1-y^2}}, \quad -1 \leq y \leq 1
\end{equation*}

\section{Inverse CDF Method}




















\section{Alpha-stable Distribution}

\section{Conclusion}

\begin{thebibliography}{99}
  \bibitem{labsheet} Cambridge University Engineering Department. 
                      \textit{Random Variables and Random Number Generation Lab Sheet}. 2025.
  \bibitem{Multinomial} S Sinharay. \textit{Discrete Probability Distributions}. ETS, Princeton, NJ, USA, 2010
\end{thebibliography}



\end{document}