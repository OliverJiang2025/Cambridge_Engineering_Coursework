\documentclass{article}
\usepackage{amsmath}
\usepackage{pdfpages}
\usepackage{amsfonts}
\usepackage{amssymb}
\usepackage{graphicx}
\graphicspath{{../Figures/}}
\usepackage{tikz}
\usepackage{amssymb}
\usepackage{circuitikz}
\usepackage{setspace}
\usepackage{gensymb}
\usepackage{hyperref}
\usepackage{subcaption}
\hypersetup{colorlinks=true, linkcolor=blue, urlcolor=blue}
\doublespacing{}
\usepackage{wrapfig}
\textwidth 15.5cm \oddsidemargin 0cm \topmargin -1cm \textheight
24cm \footskip 1.5cm \usepackage{epsfig}
\usepackage{amsmath,graphicx,psfrag,pstcol}
\usepackage{hyperref}
\def\n{\noindent}
\def\u{\underline}
\def\hs{\hspace}
\newcommand{\thrfor}{.^{\displaystyle .} .}
\newcommand{\bvec}[1]{{\bf #1}}
\usepackage{listings} 
\usepackage{xcolor}



\begin{document}
\includepdf{IIA_Lab_FeedbackSheet_2025.pdf}

\section{Lab Worksheet}
\subsection{Simplified Aircraft Model}
Transfer function = $\frac{10}{s^2+10s}$
\newline
num = [10], den = [1,10,0]
\subsection{Modelling Manual Control}
Controller transfer function = $ke^{-Ds}$
\newline
k = 1.56, D = 0.57, Phase margin = {30\degree}
\newline
Amount of extra time delay which can be tolerated = 0.338s
\subsection{Pilot Induced Oscillation}
Period of oscillation (observed) = 4.44s
\newline
Period of oscillation (theoretical) = 4.488s
\subsection{Sinusoidal Disturbances}
Maximum stabilising gain = 0.422
\newline
Gain at 0.66 Hz = 9dB   
\newline
Phase at 0.66 Hz = -240\degree
\subsection{Unstable Aircraft}
Fastest pole at T = 0.6s
\subsection{Autopilot with Proportion Control}
Proportional gain $K_C$ = 16, Period of oscillation $T_C$ = 2s
\subsection{Autopilot with PID Control}
Transfer function of PID controller = $K_P(1+\frac{1}{sT_i}+sT_d)$
\newline
PID constants: $K_P$ = 9.6, $T_i$ = 1, $T_d$ = 0.25
\newline
Adjusted value of $T_d$ = 0.35
\subsection{Integrator Wind-up}
Integrator bound Q = 0.2
\newpage
\section{Lab Report}
\subsection{(§2.2 Modelling Manual Control) Nyquist diagram (from Bode diagram) for
controller in series with plant.}
From Figure \ref{fig:fig2}, in which the upper graph is the magnitude plot 
while the lower graph is the phase plot. Information contained in this 
two plot are combined to give the shape of Nyquist Diagram of $K(j\omega)G(j\omega)$,
which is given in Figure \ref{fig:nyq}.

\begin{wrapfigure}{r}{0.4\textwidth} 
    \centering
    \includegraphics[width=\linewidth, keepaspectratio]{NyqDia.png}
    \caption{Step response of manual control}
    \label{fig:nyq}
  \end{wrapfigure}
  
It is clear that the Nyquist Diagram behaves in a spiral shape 
approaching the origin point of the Argand Diagram. 
It is due to $e^{-j\omega D}$ term which rises an exponential decay of 
magnitude while keeping the curve "rotating" as 
$\omega \to \infty$.
Besides, it is worth noting that the Nyquist diagram does not encircle $(-1,j0)$ 
according to the magnitude plot in Figure \ref{fig:fig2}. 
By Nyquist Stability Criteria, this closed loop system is stable.


\subsubsection{(§2.2 Modelling Manual Control) Are you using any integral action? Give a brief
explanation. What does this imply about the accuracy of the model of the human
controller?}
As shown by Figure \ref{fig:fig3}, it is clear that an integral action was used. 
Starting from around 4.7s, the orange $u(t)$ curve, which represents the manual 
control signal, maintains at around 5 to attempting to stabilize $e(t)$ to 0. 
This is an integral control because both the $e(t)$ and $\frac{de(t)}{dt}$ 
are zero afterwards. $u(t) = k_i \int^t_0 e(s)ds$ for some $k_i$ here.


\subsection{(§2.3 Pilot Induced Oscillation) Explain the oscillation of the feedback loop. How
does your observed period of oscillation compare to the theoretical prediction?}
Due to the poorly designed controller, there is a large phase lag at a certain
frequency. When the phase lag reaches $-\frac{\pi}{2}$, the closed loop becomes
marginally stable, and oscillations occur.
This frequency at which the aircraft oscillates can be found by looking at the
Bode plot in Figure \ref{fig:fig6}. It is clear that when the frequency is around
1.4 rad/s, the phase lag reaches $-\frac{\pi}{2}$ and the magnitude is around 0dB.
From this, the theoretical period of oscillation can be calculated as: 
$T = \frac{2\pi}{\omega} = \frac{2\pi}{1.4} = 4.488s$.
\newline
The observed period of oscillation from the simulation is 4.44s, which is very close to
the theoretical prediction of 4.488s.


\subsection{(§2.3 Pilot Induced Oscillation) Can you give a rough guideline to the control
designer to make PIO less likely?}
For designers to make PIO less likely, they should make sure the phase margin
is sufficiently large at the frequency which gives the magniude to be 0dB.
According to relevant literatures\cite{flyquality}, the average
phase gradient is more important to avoid PIO than the phase margin itself.
This gradient must be small to allow the pilot to react to the oscillations.




\subsection{(§2.4 Sinusoidal Disturbances) Was your manual input able to reduce the error
(as compared to providing no input)?}
It is very hard to keep the aircraft stable with or without manual control. By comparing
Figure \ref{fig:fig8} and Figure \ref{fig:fig10}, it is clear that the error $e(t)$ barely
reduced by manual control, even under manual control with a small movement.


\subsection{(§2.5 An Unstable Aircraft) Nyquist diagram for G2(s).}
\begin{figure}[h]
    \centering
    \includegraphics[width=0.5\textwidth]{nyqdia2.png}
    \caption{Nyquist diagram for G2(s)}
    \label{fig:nyqdia2}
\end{figure}



\subsection{(§2.5 An Unstable Aircraft) Explain, using the Nyquist criterion, why the 
feedback system is stable with a proportional gain greater than 0.5.}
For a system with proportional gain $K$, the Nyquist stability criterion states that
the closed-loop system is stable if the Nyquist plot of $G(s)$ does not encircle
$(-\frac{1}{K}, j0)$. From Figure \ref{fig:nyqdia2}, it is clear that when $K > 0.5$,
the system is stable as the plot intersect the negative real axis at (2,0).





\subsection{(§2.5 An Unstable Aircraft) Sketch of a Nyquist diagram for G2(s). with a small
time delay D.}
\begin{figure}[h]
    \centering
    \includegraphics[width=0.5\textwidth]{nyqdia3.png}
    \caption{Nyquist diagram for G2(s) with time delay D}
    \label{fig:nyqdia3}
\end{figure}

\subsection{(§3.4 Integrator Wind-up) Explain how you calculated the bound on Q.}
The time domain expression of the PID controller is:
\begin{equation*}
    u(t) = K_P \left( e(t) + \frac{1}{T_i} \int_0^t e(\tau) d\tau + T_d \frac{de(t)}{dt} \right)
\end{equation*}
And the bound Q is defined to be the maximum value of $\int_0^t e(\tau) d\tau$.
We can see that at steady state, $e(t) = 0$ and $\frac{de(t)}{dt} = 0$, so the only term
that contributes to $u(t)$ is the integral term. Therefore, we have:
$u(t) = \frac{k_P}{T_i} \int_0^t e(\tau) d\tau$, and Q = $\frac{T_i}{K_P} u_{max} \approx 0.2$.


\newpage
\appendix
\section{Link to Jupyter Notebook}
This \href{https://github.com/OliverJiang2025/3F1_Lab_Flight_Control.git}{link} goes to the worked Jupyter Notebook finished on the lab session.

\section{Figures from Lab Session}

\begin{figure}[h]
    \centering
    % 第1行:Figure1-3.png
    \begin{subfigure}[b]{0.28\textwidth}  % 缩小子图宽度,预留间距
        \centering
        \includegraphics[width=\linewidth]{Figure1.png}  % 带.png后缀
        \caption{Supplementary Figure 1}
        \label{fig:fig1}
    \end{subfigure}
    \hfill
    \begin{subfigure}[b]{0.28\textwidth}
        \centering
        \includegraphics[width=\linewidth]{Figure2.png}
        \caption{Supplementary Figure 2}
        \label{fig:fig2}
    \end{subfigure}
    \hfill
    \begin{subfigure}[b]{0.28\textwidth}
        \centering
        \includegraphics[width=\linewidth]{Figure3.png}
        \caption{Supplementary Figure 3}
        \label{fig:fig3}
    \end{subfigure}

    % 第2行:Figure4-6.png
    \vspace{0.3cm}  % 紧凑垂直间距
    \begin{subfigure}[b]{0.28\textwidth}
        \centering
        \includegraphics[width=\linewidth]{Figure4.png}
        \caption{Supplementary Figure 4}
        \label{fig:fig4}
    \end{subfigure}
    \hfill
    \begin{subfigure}[b]{0.28\textwidth}
        \centering
        \includegraphics[width=\linewidth]{Figure5.png}
        \caption{Supplementary Figure 5}
        \label{fig:fig5}
    \end{subfigure}
    \hfill
    \begin{subfigure}[b]{0.28\textwidth}
        \centering
        \includegraphics[width=\linewidth]{Figure6.png}
        \caption{Supplementary Figure 6}
        \label{fig:fig6}
    \end{subfigure}

    % 第3行:Figure7-8.png + 空占位
    \vspace{0.3cm}
    \begin{subfigure}[b]{0.28\textwidth}
        \centering
        \includegraphics[width=\linewidth]{Figure7.png}
        \caption{Supplementary Figure 7}
        \label{fig:fig7}
    \end{subfigure}
    \hfill
    \begin{subfigure}[b]{0.28\textwidth}
        \centering
        \includegraphics[width=\linewidth]{Figure8.png}
        \caption{Supplementary Figure 8}
        \label{fig:fig8}
    \end{subfigure}
    \hfill
    % 空占位(保持3列结构整齐)
    \begin{subfigure}[b]{0.28\textwidth}
        \centering
        \mbox{}  % 空内容,不影响排版
    \end{subfigure}

    \caption{Supplementary figures 1--8 (continued on next page).}
    \label{fig:app-page1}
\end{figure}

% ---------------------- 第2页:7张图(Figure9-15.png) ----------------------
\begin{figure}[h]
    \centering
    % 第1行:Figure9-11.png
    \begin{subfigure}[b]{0.28\textwidth}
        \centering
        \includegraphics[width=\linewidth]{Figure9.png}
        \caption{Supplementary Figure 9}
        \label{fig:fig9}
    \end{subfigure}
    \hfill
    \begin{subfigure}[b]{0.28\textwidth}
        \centering
        \includegraphics[width=\linewidth]{Figure10.png}
        \caption{Supplementary Figure 10}
        \label{fig:fig10}
    \end{subfigure}
    \hfill
    \begin{subfigure}[b]{0.28\textwidth}
        \centering
        \includegraphics[width=\linewidth]{Figure11.png}
        \caption{Supplementary Figure 11}
        \label{fig:fig11}
    \end{subfigure}

    % 第2行:Figure12-14.png
    \vspace{0.3cm}
    \begin{subfigure}[b]{0.28\textwidth}
        \centering
        \includegraphics[width=\linewidth]{Figure12.png}
        \caption{Supplementary Figure 12}
        \label{fig:fig12}
    \end{subfigure}
    \hfill
    \begin{subfigure}[b]{0.28\textwidth}
        \centering
        \includegraphics[width=\linewidth]{Figure13.png}
        \caption{Supplementary Figure 13}
        \label{fig:fig13}
    \end{subfigure}
    \hfill
    \begin{subfigure}[b]{0.28\textwidth}
        \centering
        \includegraphics[width=\linewidth]{Figure14.png}
        \caption{Supplementary Figure 14}
        \label{fig:fig14}
    \end{subfigure}

    % 第3行:Figure15.png(居中,避免单独一列溢出)
    \vspace{0.3cm}
    \begin{subfigure}[b]{0.28\textwidth}  % 与其他子图宽度一致
        \centering
        \includegraphics[width=\linewidth]{Figure15.png}
        \caption{Supplementary Figure 15}
        \label{fig:fig15}
    \end{subfigure}
    % 左右空占位,让第15张图居中
    \hfill
    \begin{subfigure}[b]{0.28\textwidth}
        \centering
        \mbox{}
    \end{subfigure}
    \hfill
    \begin{subfigure}[b]{0.28\textwidth}
        \centering
        \mbox{}
    \end{subfigure}

    \caption{Supplementary figures 9--15 (continued from previous page).}
    \label{fig:app-page2}
\end{figure}




\newpage
\begin{thebibliography}{99}

    \bibitem{cued} EIETL, CUED.
    \newblock \textit{3F1 Flight Control Lab Guide}.
    \bibitem{flyquality} 
    U.S. Department of Defense.
    \newblock \textit{Flying Qualities of Piloted Aircraft}.
    \newblock MIL-STD-1797A.
    \newblock Department of Defense, Washington, D.C., Jan 1990.
    \newblock (Military Standard).


\end{thebibliography}

\vspace*{1cm}



\end{document}

