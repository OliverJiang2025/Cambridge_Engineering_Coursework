\documentclass{article}
\usepackage{amsmath}
\usepackage{pdfpages}
\usepackage{amsfonts}
\usepackage{amssymb}
\usepackage{graphicx}
\usepackage{tikz}
\usepackage{amssymb}
\usepackage{circuitikz}
\usepackage{setspace}
\usepackage{gensymb}
\usepackage{hyperref}
\usepackage{subcaption}
\hypersetup{colorlinks=true, linkcolor=blue, urlcolor=blue}
\doublespacing{}
\usepackage{wrapfig}
\textwidth 15.5cm \oddsidemargin 0cm \topmargin -1cm \textheight
24cm \footskip 1.5cm \usepackage{epsfig}
\usepackage{amsmath,graphicx,psfrag,pstcol}
\usepackage{hyperref}
\def\n{\noindent}
\def\u{\underline}
\def\hs{\hspace}
\newcommand{\thrfor}{.^{\displaystyle .} .}
\newcommand{\bvec}[1]{{\bf #1}}
\usepackage{listings} 
\usepackage{xcolor}



\begin{document}
\includepdf{IIA_Lab_FeedbackSheet_2020.pdf}

\section{Introduction}
This lab activity investigates laser characteristics, 
including light-current characteristics and output spectrum. 
The goal is to understand the behavior of lasers under different 
conditions and analyze their performance based on the measured data.

\section{Part I: Measurement of Light-Current Characteristics}

The apparatus used for this measurement includes a laser diode, 
a driver circuit, a photodiode for measuring output power, 
and a transimpedance amplifier to quantify output power by voltage output.
The raw data collected from the experiment 
will be analyzed to determine some key parameters of the laser, and 
the circuits used in the experiment. The raw data will be available in appendix.


\begin{figure}[h]
    \centering
    \includegraphics[width=1.0\textwidth]{Circuit_diagram.png}
    \caption{Circuit Diagram}
    \label{fig:cctdiagram}
\end{figure}

\subsection{Laser Theory}
According to laser theory, photon emission becomes stimulated
when the gain exceeds the losses in the laser cavity. This could 
be observed in the light-current (L-I) characteristics of a laser, where
the output power increases significantly once the threshold current is reached.
The expectation of the L-I curve is that it will show a linear increase in output power
after the threshold current, with a steep slope indicating efficient lasing action.

As given in the lab handout, this threshold current is between 40 and 55 mA, 
while the actual threshold current observed from the raw data is around 75 mA. 
This could be reasonable, because the operation of laser diode
is very temperature sensitive. The threshold current can increase significantly 
with temperature.

\subsection{Data Analysis}
As shown in Figure \ref{fig:plot}, the plot of the raw data
of L-I curve shows a clear 
threshold behavior. 
Below the threshold current, the light output is negligible. This is because the 
gain is not sufficient to overcome the losses in the cavity. 
the laser is not lasing, and only spontaneous emission occurs.
Above the threshold current, the output power increases rapidly, indicating that
the laser is lasing and the gain exceeds the losses. 
The large slope of the L-I curve above the threshold can provide 
insights into the efficiency of the laser, 
with a steeper slope indicating higher efficiency in 
converting electrons to photons.

\begin{figure}[h]
    \centering
    \includegraphics[width=0.8\textwidth]{Figure_1.png}
    \caption{Plot and linear fit of raw data}
    \label{fig:plot}
\end{figure}

While the other plot shows relation of voltage across the laser diode and
the current through the laser diode. The V-I relation can be derived 
from the equivalent circuit of the laser diode, as shown in Figure \ref{fig:eqcct}.


\begin{figure}[h]
    \centering
    \includegraphics[width=1.0\textwidth]{Equivalent_circuit.png}
    \caption{Equivalent Circuit of Laser Diode}
    \label{fig:eqcct}
\end{figure}

There is a non-negligible stray series resistance in the laser diode,
which contributes to the linear
increase in voltage across the laser diode as the current increases.

The relation could be expressed as:
\begin{equation*}
    V = V_{th} + I \cdot R_s
\end{equation*}

Where V and I are the voltage across the laser diode and the current through the laser diode,
respectively. $V_{th}$ is the threshold voltage, and $R_s$ is the stray series resistance.

By the linear fit of the V-I curve, 
we can determine the threshold voltage and the stray series resistance.
$V_{th} \approx 1.28 V$, and $R_s \approx 21.145 \Omega$, whose error analysis 
will be carried out later.

From the circuit diagram given, the stray series resistance is 26 $\Omega$,
which is close to the value obtained from the linear fit, with an error of 
$17.3\%$. This error could be attributed to the measurement errors, 
and the non-ideal behavior of the laser diode.

With the threshold value of perfect diode, we could estimate the lasing
wavelength of the laser diode $\lambda_0$ by the following equation:

\begin{equation*}
    \lambda_0 = \frac{hc}{eV_{th}}
\end{equation*}

The value of wavelength is 968.39nm, which is within infrared range.
This calculation agrees to the observation during the lab. There was
no visible light emitted from the laser diode. 



\subsection{Error Analysis}

To carry out error analysis on gradient of intercept of a linear fit,
a built-in module in python, \texttt{scipy.odr}, is utilized. 
The measurement is modelled to have a $3\%$ system error and corresponding 
random error caused by human reading. The total error is assumed to be
geometric mean of the two error sources. 

The linear fit with error bar is shown in Figure \ref{fig:VI}.

\begin{figure}[h]
    \centering
    \includegraphics[width=0.8\textwidth]{VI_fit.png}
    \caption{V-I linear fit with error analysis}
    \label{fig:VI}
\end{figure}

The stary resistance is $1000 \times \text{gradient}$, with value of 
21.145 $\Omega$, and an error of $\pm 0.3861 \Omega$. It is obvious that
the real value of the stray resistance is not with in the error range of this 
linear fit. There might be other non-ideal imperfections that affected the 
measurement. For example, it could be effect of temperature or extra parasitic effect
in the circuit.

The intercept corresponds to the threshold voltage of the laser diode. It 
has a value of 1.2812V with error of $\pm0.0221$V,
or a percentage error of $\pm1.72\%$. This error would propagate
to the calculation of wavelength of light emitted.

The wavelength is calculated by $\lambda_0 = \frac{hc}{eV_{th}}$, in 
which $h$, $c$, $e$ are all constants. The only source of error 
comes from $V_{th}$, which gives a percentage error of $\pm1.72\%$.
This is an absolute error of $\pm16.66$nm.

Thus, the stray resistance is measured to be $21.145 \pm 0.3861 \Omega$.
The wavelength of light emitted is measured to be $968.39 \pm 16.66$nm.


\section{Part II: Measurement of laser output spectrum}
In thie part, the output spectrum of the laser diode 
is measured using an optical spectrum analyzer. Data is collected
for both stimulated and under-stimulated conditions, 
and the results are analyzed to understand the spectral 
characteristics of the laser output.

\subsection{Effective Refraction Index of the Semiconductor Material}
Within the cavity of the laser, the light forms a standing wave,
with a wavelength $\lambda_g = \frac{\lambda}{n}$. 
Where $\lambda$ is the wavelength of the light in vacuum, 
and $n$ is the effective refraction index of the semiconductor material.
This wavelength should satisfy the resonance condition of the cavity, which is given by:
\begin{equation*}
    m  \frac{\lambda_g}{2} = L
\end{equation*}
Where m is an integer representing the mode number, and L is the length of the cavity. 
By rearranging the equation, we can express the wavelength in vacuum as:
\begin{equation*}
    \lambda = \frac{2nL}{m} \to f = \frac{mc}{2nL}
\end{equation*}

The change in frequency is:
\begin{equation*}
    \delta f = \frac{c}{2nL}
\end{equation*}

From $f = \frac{c}{\lambda}$, $\delta f = \frac{c}{\lambda^2} \delta \lambda$, 
we can derive the change in wavelength as:
\begin{equation*}
    \delta \lambda = \frac{\lambda^2}{2nL}
\end{equation*}

Thus, the effective refraction index can be calculated as:
\begin{equation*}
    n = \frac{\lambda^2}{2L \Delta \lambda}
\end{equation*}

\subsection{Measurement of Laser Output Spectrum}
The observed output spectrum of the laser diode
shows multiple peaks, which correspond to different longitudinal modes of the laser cavity.
Figure \ref{fig:four_images} and Figure \ref{fig:unfour_images} in the appendix
show the output spectrum under stimulated and under-stimulated conditions, 
respectively. The collected data is shown in Figure \ref{fig:specdata}.
\begin{figure}[h]
    \centering
    \includegraphics[width=1.0\textwidth]{spec_data.png}
    \caption{Spectrum Data}
    \label{fig:specdata}
\end{figure}


\subsection{Discussion and Analysis}
By comparing the two sets of data, it is clear that 
the only discrepancy is the 3dB width of the spectrum. 
The 3dB width of the spectrum under stimulated condition is narrower 
than that under under-stimulated condition. This is 
expected because the stimulated photon emission 
leads to a more concentrated spectrum,
while spontaneous emission results in a broader spectrum 
due to the random nature of photon emission.

The center wavelengths of the two spectra are very close, 
which indicates that the lasing wavelength range is consistent 
regardless of the stimulation condition.

At last, the ripple width corresponds to $\Delta \lambda$ 
in the equation for effective refraction index. As the two 
measurements use the same laser diode, the ripple widths are 
expected to be the same.

The effective refraction index could be calculated as:
\begin{equation*}
    n = \frac{\lambda_{center}^2}{2L \Delta \lambda} = 3.351
\end{equation*}
by taking $L = 300 \mu m$. This value of refractive index is 
within the typical range of semiconductor.

\subsection{Error Analysis}

The calculation of refractive index has two sources of error:
$\Delta \lambda$ and $\lambda_{center}^2$. Thus, the error of refractive 
index should include both sources:
\begin{equation*}
    U_n = \sqrt{(\frac{\partial n}{\partial \Delta \lambda}U_{\Delta \lambda})^2
            + (\frac{\partial n}{\partial \lambda_{center}^2}U_{\lambda_{center}^2})^2}
\end{equation*}
in which $U_x$ is the absolute error of measurement on $x$.

For each reading of wavelength from the spectrum analyzer, there is a
random error of $U_0$ = 0.00005$\mu$m = 0.05nm. 

By measuring $\Delta \lambda$, two readings were taken to measure width of 
15 ripples. The error is diluted by a factor of 15:
$U_{\Delta \lambda} = \frac{2U_0}{15} = 6.67$pm.

The percentage error of $\lambda_{center}^2$ should be twice the percentage 
error of $\lambda_{center}$. $\%U_{\lambda_{center}} = \frac{0.05\text{nm}}{1.5565\mu\text{m}}$
= 3.2\text{ppm}.
The absolute error of $\lambda_{center}^2$ is then 
$(1.5565\mu m)^2 \times 2 \times 3.2 \text{ppm} = 1.55 \times 10^{-17}$m.

The partial differentials are found by:
\begin{equation*}
    \frac{\partial n}{\partial \Delta \lambda} = 
    -\frac{\lambda_{center}^2}{2L (\Delta \lambda)^2} = -2.78 \times 10^{9}m,
    \frac{\partial n}{\partial \lambda_{center}^2} =
    \frac{1}{2L \Delta \lambda} = 1.38 \times 10^{12}m
\end{equation*}

The total error of refractive index is:
\begin{equation*}
    U_n = \sqrt{(2.78 \times 10^9 \times 6.67 \times 10^{-12})^2+
                (1.38 \times 10^{12} \times 1.55 \times 10^{-17})^2}
                = 0.0185
\end{equation*}

Thus, the measured value of refractive index is $3.351\pm0.0185$.


\section{Conclusion}

In this experiment, the electro-optical characteristics of a 
laser diode were successfully characterized. 
The Light-Current (L-I) analysis confirmed the transition 
from spontaneous to stimulated emission. 
Through linear regression of the V-I characteristics, 
the threshold voltage and the stray series resistance 
are determined. 
Based on the threshold voltage, 
the bandgap emission wavelength was estimated, 
confirming operation in the infrared region.

Spectral analysis demonstrated the difference of 
3dB bandwidth under stimulated conditions compared to spontaneous emission. 
By analyzing the longitudinal mode spacing, 
the effective refractive index of the semiconductor material was 
calculated. 

Overall, the experimental results align well with theoretical 
expectations for semiconductor laser diodes, 
accounting for temperature sensitivity and parasitic circuit effects.

\section{Appendix}

\begin{figure}[h]
    \centering
    \includegraphics[width=1.0\textwidth]{rawdata.png}
    \caption{Raw Data}
    \label{fig:rawdata}
\end{figure}



\begin{figure}[htbp]
    \centering
    \begin{subfigure}[b]{0.4\textwidth}
        \centering
        \includegraphics[width=\linewidth]{stimulated_view.jpg} 
        \caption{}
        \label{fig:sub1}
    \end{subfigure}
    \hfill 
    \begin{subfigure}[b]{0.4\textwidth}
        \centering
        \includegraphics[width=\linewidth]{stimulated_centre.jpg} 
        \caption{}
        \label{fig:sub2}
    \end{subfigure}
    
    \par\bigskip 
    
    \begin{subfigure}[b]{0.4\textwidth}
        \centering
        \includegraphics[width=\linewidth]{stimulated_ripple.jpg} 
        \caption{}
        \label{fig:sub3}
    \end{subfigure}
    \hfill
    \begin{subfigure}[b]{0.4\textwidth}
        \centering
        \includegraphics[width=\linewidth]{stimulated_3dBwidth.jpg} 
        \caption{}
        \label{fig:sub4}
    \end{subfigure}
    
    \caption{Stimulated Output Spectrum of the Laser Diode}
    \label{fig:four_images}
\end{figure}




\begin{figure}[htbp]
    \centering
    \begin{subfigure}[b]{0.4\textwidth}
        \centering
        \includegraphics[width=\linewidth]{under_view.jpg} 
        \caption{}
        \label{fig:unsub1}
    \end{subfigure}
    \hfill 
    \begin{subfigure}[b]{0.4\textwidth}
        \centering
        \includegraphics[width=\linewidth]{under_center.jpg} 
        \caption{}
        \label{fig:unsub2}
    \end{subfigure}
    
    \par\bigskip 
    
    \begin{subfigure}[b]{0.4\textwidth}
        \centering
        \includegraphics[width=\linewidth]{under_ripple.jpg} 
        \caption{}
        \label{fig:unsub3}
    \end{subfigure}
    \hfill
    \begin{subfigure}[b]{0.4\textwidth}
        \centering
        \includegraphics[width=\linewidth]{under_3dBwidth.jpg} 
        \caption{}
        \label{fig:nusub4}
    \end{subfigure}
    
    \caption{Under Stimulated Output Spectrum of the Laser Diode}
    \label{fig:unfour_images}
\end{figure}

\end{document}

