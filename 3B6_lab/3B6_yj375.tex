\documentclass{article}
\usepackage{amsmath}
\usepackage{pdfpages}
\usepackage{amsfonts}
\usepackage{amssymb}
\usepackage{graphicx}
\usepackage{tikz}
\usepackage{amssymb}
\usepackage{circuitikz}
\usepackage{setspace}
\usepackage{gensymb}
\usepackage{hyperref}
\usepackage{subcaption}
\hypersetup{colorlinks=true, linkcolor=blue, urlcolor=blue}
\doublespacing{}
\usepackage{wrapfig}
\textwidth 15.5cm \oddsidemargin 0cm \topmargin -1cm \textheight
24cm \footskip 1.5cm \usepackage{epsfig}
\usepackage{amsmath,graphicx,psfrag,pstcol}
\usepackage{hyperref}
\def\n{\noindent}
\def\u{\underline}
\def\hs{\hspace}
\newcommand{\thrfor}{.^{\displaystyle .} .}
\newcommand{\bvec}[1]{{\bf #1}}
\usepackage{listings} 
\usepackage{xcolor}



\begin{document}
\includepdf{IIA_Lab_FeedbackSheet_2020.pdf}

\section{Introduction}
This lab activity investigates laser characteristics, 
including light-current characteristics and output spectrum. 
The goal is to understand the behavior of lasers under different 
conditions and analyze their performance based on the measured data.

\section{Part I: Measurement of Light-Current Characteristics}

The apparatus used for this measurement includes a laser diode, 
a driver circuit, a photodiode for measuring output power, 
and a transimpedance amplifier to quantify output power by voltage output.
The raw data collected from the experiment 
will be analyzed to determine some key parameters of the laser, and 
the circuits used in the experiment. The raw data will be available in appendix.


\begin{figure}[h]
    \centering
    \includegraphics[width=1.0\textwidth]{Circuit_diagram.png}
    \caption{Circuit Diagram}
    \label{fig:cctdiagram}
\end{figure}

\subsection{Laser Theory}
According to laser theory, photon emission becomes stimulated
when the gain exceeds the losses in the laser cavity. This could 
be observed in the light-current (L-I) characteristics of a laser, where
the output power increases significantly once the threshold current is reached.
The expectation of the L-I curve is that it will show a linear increase in output power
after the threshold current, with a steep slope indicating efficient lasing action.

As given in the lab handout, this threshold current is between 40 and 55 mA, 
while the actual threshold current observed from the raw data is around 75 mA. 
This could be reasonable, because the operation of laser diode
is very temperature sensitive. The threshold current can increase significantly 
with temperature.

\subsection{Data Analysis}
As shown in Figure \ref{fig:plot}, the plot of the raw data
of L-I curve shows a clear 
threshold behavior. 
Below the threshold current, the light output is negligible. This is because the 
gain is not sufficient to overcome the losses in the cavity. 
the laser is not lasing, and only spontaneous emission occurs.
Above the threshold current, the output power increases rapidly, indicating that
the laser is lasing and the gain exceeds the losses. 
The large slope of the L-I curve above the threshold can provide 
insights into the efficiency of the laser, 
with a steeper slope indicating higher efficiency in 
converting electrons to photons.

\begin{figure}[h]
    \centering
    \includegraphics[width=0.8\textwidth]{Figure_1.png}
    \caption{Plot and linear fit of raw data}
    \label{fig:plot}
\end{figure}

While the other plot shows relation of voltage across the laser diode and
the current through the laser diode. The V-I relation can be derived 
from the equivalent circuit of the laser diode, as shown in Figure \ref{fig:eqcct}.


\begin{figure}[h]
    \centering
    \includegraphics[width=1.0\textwidth]{Equivalent_circuit.png}
    \caption{Equivalent Circuit of Laser Diode}
    \label{fig:eqcct}
\end{figure}

There is a non-negligible stray series resistance in the laser diode,
which contributes to the increase in voltage across the laser diode as the current increases.

The relation could be expressed as:
\begin{equation*}
    V = V_{th} + I \cdot R_s
\end{equation*}

Where V and I are the voltage across the laser diode and the current through the laser diode,
respectively. $V_{th}$ is the threshold voltage, and $R_s$ is the stray series resistance.

By the linear fit of the V-I curve, 
we can determine the threshold voltage and the stray series resistance.
$V_{th} \approx 1.262 V$, and $R_s \approx 21.505 \Omega$.

From the circuit diagram given, the stray series resistance is 26 $\Omega$,
which is close to the value obtained from the linear fit, with an error of 
$17.3\%$. This error could be attributed to the measurement errors, 
and the non-ideal behavior of the laser diode.

With the threshold value of perfect diode, we could estimate the lasing
wavelength of the laser diode $\lambda_0$ by the following equation:

\begin{equation*}
    \lambda_0 = \frac{hc}{eV_{th}}
\end{equation*}

The value of wavelength is 983.12nm, which is within infrared range.
This calculation agrees to the observation during the lab, there was
no visible light emitted from the laser diode. 



\subsection{Error Analysis}






\section{Part II: Measurement of laser output spectrum}



\begin{figure}[htbp]
    \centering
    \begin{subfigure}[b]{0.45\textwidth}
        \centering
        \includegraphics[width=\linewidth]{stimulated_view.jpg} 
        \caption{}
        \label{fig:sub1}
    \end{subfigure}
    \hfill 
    \begin{subfigure}[b]{0.45\textwidth}
        \centering
        \includegraphics[width=\linewidth]{stimulated_centre.jpg} 
        \caption{}
        \label{fig:sub2}
    \end{subfigure}
    
    \par\bigskip 
    
    \begin{subfigure}[b]{0.45\textwidth}
        \centering
        \includegraphics[width=\linewidth]{stimulated_ripple.jpg} 
        \caption{}
        \label{fig:sub3}
    \end{subfigure}
    \hfill
    \begin{subfigure}[b]{0.45\textwidth}
        \centering
        \includegraphics[width=\linewidth]{stimulated_3dBwidth.jpg} 
        \caption{}
        \label{fig:sub4}
    \end{subfigure}
    
    \caption{}
    \label{fig:four_images}
\end{figure}




\begin{figure}[htbp]
    \centering
    \begin{subfigure}[b]{0.45\textwidth}
        \centering
        \includegraphics[width=\linewidth]{under_view.jpg} 
        \caption{}
        \label{fig:unsub1}
    \end{subfigure}
    \hfill 
    \begin{subfigure}[b]{0.45\textwidth}
        \centering
        \includegraphics[width=\linewidth]{under_center.jpg} 
        \caption{}
        \label{fig:unsub2}
    \end{subfigure}
    
    \par\bigskip 
    
    \begin{subfigure}[b]{0.45\textwidth}
        \centering
        \includegraphics[width=\linewidth]{under_ripple.jpg} 
        \caption{}
        \label{fig:unsub3}
    \end{subfigure}
    \hfill
    \begin{subfigure}[b]{0.45\textwidth}
        \centering
        \includegraphics[width=\linewidth]{under_3dBwidth.jpg} 
        \caption{}
        \label{fig:nusub4}
    \end{subfigure}
    
    \caption{}
    \label{fig:unfour_images}
\end{figure}






































\section{Conclusion}

\section{Appendix}

\begin{figure}[h]
    \centering
    \includegraphics[width=1.0\textwidth]{rawdata.png}
    \caption{Raw Data}
    \label{fig:rawdata}
\end{figure}


\end{document}

