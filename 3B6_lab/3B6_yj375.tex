\documentclass{article}
\usepackage{amsmath}
\usepackage{pdfpages}
\usepackage{amsfonts}
\usepackage{amssymb}
\usepackage{graphicx}
\usepackage{tikz}
\usepackage{amssymb}
\usepackage{circuitikz}
\usepackage{setspace}
\usepackage{gensymb}
\usepackage{hyperref}
\usepackage{subcaption}
\hypersetup{colorlinks=true, linkcolor=blue, urlcolor=blue}
\doublespacing{}
\usepackage{wrapfig}
\textwidth 15.5cm \oddsidemargin 0cm \topmargin -1cm \textheight
24cm \footskip 1.5cm \usepackage{epsfig}
\usepackage{amsmath,graphicx,psfrag,pstcol}
\usepackage{hyperref}
\def\n{\noindent}
\def\u{\underline}
\def\hs{\hspace}
\newcommand{\thrfor}{.^{\displaystyle .} .}
\newcommand{\bvec}[1]{{\bf #1}}
\usepackage{listings} 
\usepackage{xcolor}



\begin{document}
\includepdf{IIA_Lab_FeedbackSheet_2020.pdf}

\section{Introduction}
This lab activity investigates laser characteristics, 
including light-current characteristics and output spectrum. 
The goal is to understand the behavior of lasers under different 
conditions and analyze their performance based on the measured data.

\section{Part I: Measurement of Light-Current Characteristics}

The apparatus used for this measurement includes a laser diode, 
a driver circuit, a photodiode for measuring output power, 
and a transimpedance amplifier to quantify output power by voltage output.
The raw data collected from the experiment 
will be analyzed to determine some key parameters of the laser, and 
the circuits used in the experiment. The raw data will be available in appendix.


\begin{figure}[h]
    \centering
    \includegraphics[width=1.0\textwidth]{Circuit_diagram.png}
    \caption{Circuit Diagram}
    \label{fig:cctdiagram}
\end{figure}

\subsection{L-I Curve Analysis}
According to laser theory, photon emission becomes stimulated
when the gain exceeds the losses in the laser cavity. This could 
be observed in the light-current (L-I) characteristics of a laser, where
the output power increases significantly once the threshold current is reached.
The expectation of the L-I curve is that it will show a linear increase in output power
after the threshold current, with a steep slope indicating efficient lasing action.

As given in the lab handout, this threshold current is between 40 and 55 mA.
\begin{figure}[h]
    \centering
    \includegraphics[width=1.0\textwidth]{Figure_1.png}
    \caption{Plot and linear fit of raw data}
    \label{fig:plot}
\end{figure}




















\section{Part II: Measurement of laser output spectrum}

\section{Conclusion}

\section{Appendix}

\begin{figure}[h]
    \centering
    \includegraphics[width=1.0\textwidth]{rawdata.png}
    \caption{Raw Data}
    \label{fig:rawdata}
\end{figure}


\end{document}

