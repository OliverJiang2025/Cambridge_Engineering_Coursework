\documentclass[11pt]{article}[times]

\topmargin 0.0cm 
\oddsidemargin 0.0in 
\evensidemargin0.0in
\textheight 22cm 
\textwidth  17cm 
\headheight 0in 
\headsep 0in
\parindent0in


\usepackage{amsmath}
\usepackage{amsfonts}
\usepackage{amssymb}
\usepackage{graphicx}
\graphicspath{{./Figures/}}
\usepackage{tikz}
\usepackage{amssymb}
\usepackage{circuitikz}
\usepackage{setspace}
\usepackage{pdfpages}
\begin{document}


\section*{Experiment 1: C-V Measurements}
\begin{figure}[htbp]
  \includegraphics[width=0.8\textwidth]{Figure_1.png} 
  \centering
  \caption{Circuit Diagram for CV Measurements} 
  \label{fig:fig1} 
\end{figure}
The circuits shown in Figure \ref{fig:fig1}
is used to conduct the CV Measurements.

The open-loop frequency of the oscillator 
$f_0 = 1629.9 \pm 0.05 \text{kHz}$.

The "10 pF" capacitor $C_{10} = 10.48 \pm 0.005 pF$.

With $C_{10}$ connected, the frequency $f_{10} = 1467.4 \pm 0.05 \text{kHz}$.

From theories of oscillators, 
the circuit capacitance 
$C_0 = \frac{C_{10}}{\big(\frac{f_0^2}{f_{10}^2}\big)-1} = 44.84 \pm 0.054 \text{pF}$.
The error $\Delta C_0$ is found by:
$\Delta C_0 \approx |\frac{\partial C_{0}}{\partial C_{10}}| \Delta C_{10} + |\frac{\partial C_{0}}{\partial C_{0}}| \Delta f_{0} + |\frac{\partial C_{0}}{\partial f_{10}}| \Delta f_{10}$.

In this experiment, the total 
capacitance $C_r$ is consisted of the
stray capacitance $C_s$, and capacitance
of the depletion region $C_{diode}$:
$C_r = C_s + C_{diode}$.

Where $C_s$ is constant and 
$C_{diode} = A_c (\frac{\epsilon_s e N_d}{2(V_o-V_{rev})})^{1/2}$,
$V_o = 0.5V$ is the built-in voltage. 

The bias voltage $V_{rev}$ is varied in a 
range and corresponding $C_r$ is found from 
$C_r = C_0[(\frac{f_0}{f_r})^2-1]$, where 
$f_r$ is the frequency measured on the frequency 
meter. 

The data table is shown in Figure \ref{tab:tab1} in 
appendix. 

To extrapolate $C_s$ and $N_D$ from the data, 
a scatter plot and a linear fit are applied, 
where $C_r$ is plotted against $(V_0 - V_{ref})^{-1/2}$.

Rearrangements gives: $C_r = C_s + [A_d (\frac{\epsilon_s eN_D}{2})^{1/2}] \cdot (V_0 - V_{ref})^{-1/2}$.

For the linear fit $y = kx + b$, $C_s = b$
and $N_D = \frac{2}{\epsilon_s e}(\frac{k}{A_c})^2 $.

\begin{figure}[htbp]
  \includegraphics[width=0.5\textwidth]{Figure_2.png} 
  \centering
  \caption{Linear Fit of the CV Measurements} 
  \label{fig:fig2} 
\end{figure}

With k = 6.2615, b = 2.4309, 
$C_s = 2.4309 pF$, $N_D = 8.2 \times 10^{26} m^{-3}$.

\section*{Experiment 2: I-V Measurements}

\subsection*{Reverse Bias}

\begin{figure}[htbp]
  \includegraphics[width=0.5\textwidth]{Figure_3.png} 
  \centering
  \caption{Circuit Diagram of Measuring Reverse Saturation Current} 
  \label{fig:fig3} 
\end{figure}

Figure \ref{fig:fig3} shows the circuits diagram 
for measuring reverse saturation current under
reverse bias. 

The reverse current $I_{rev}$ is related to 
reverse saturation current $I_{s}$ and 
leak resistance $R_{leak}$ by:
$I_{rev} = I_s [\exp{\frac{eV_{rev}}{\eta kT}}-1] + \frac{V_{rev}}{R_{leak}}$

\begin{figure}[htbp]
  \includegraphics[width=0.5\textwidth]{Figure_5.png} 
  \centering
  \caption{Linear Fit for Current - Voltage Measurements} 
  \label{fig:fig5} 
\end{figure}

Figure \ref{fig:fig5} shows a linear fit of
measured values of $I_{rev} - V_{rev}$.

The reverse saturation current can be estimated
by the intercept as $I_{rev} \approx - I_s$
when $V_{rev} = 0$. $I_s \approx 0.1594 \mu A$.

The leak resistance can be estimated by the gradient.
$R_{leak} \approx 59.2 k\Omega $.


\subsection*{Weak Forward Bias} 

\begin{figure}[htbp]
  \includegraphics[width=0.5\textwidth]{Figure_4.png} 
  \centering
  \caption{Circuit Diagram of Measuring Forward Current}
  \label{fig:fig4} 
\end{figure}

Figure \ref{fig:fig4} shows the circuits diagram 
for measuring current under
both weak forward bias and strong 
forward bias. 

For a Schotkky barrier diode under forward bias, 
$I_F = I_s [\exp{\frac{eV}{\eta kT}}-1]$.
Rearrangement gives 
$V_F = \eta \frac{kT}{e}[\ln(\frac{I_F}{I_s})+1]$.

$V_F$ is plotted against $(\ln(\frac{I_F}{I_s})+1)$,
shown in Figure \ref{fig:fig6}. The gradient 
equals to $\eta \frac{kT}{e}$. Under the 
lab circumstance, $\frac{kT}{e} = 25.68mV$.
$\eta = 1.47$.

\begin{figure}[htbp]
  \includegraphics[width=0.6\textwidth]{Figure_6.png} 
  \centering
  \caption{Linear Fit of Forward Voltage and $\ln(\frac{I_F}{I_s})+1$}
  \label{fig:fig6} 
\end{figure}

\subsection*{Strong Forward Bias}
In this section, a same circuit is used 
as the test in weak forward bias. 

From the lab sheet, 
$I_F = \frac{1}{r_c} (V_F - \frac{\eta kT}{e}\ln(\frac{I_F}{I_s}))$.



\begin{figure}[htbp]
  \includegraphics[width=0.6\textwidth]{Figure_7.png} 
  \centering
  \caption{Linear Fit of Forward Voltage and Forward Current}
  \label{fig:fig7} 
\end{figure}

\begin{figure}[htbp]
  \includegraphics[width=0.6\textwidth]{Figure_8.png} 
  \centering
  \caption{Linear Fit of Forward Voltage and Forward Current}
  \label{fig:fig8} 
\end{figure}

$r_{cSBD} = 2.5 \Omega$, $r_{cpn} = 0.9 \Omega$.



\appendix

\section*{Data Tables}


\begin{figure}[htbp]
  \includegraphics[width=0.8\textwidth]{Table_1.png} 
  \centering
  \caption{Data Table for CV Measurements} 
  \label{tab:tab1} 
\end{figure}



\begin{figure}[htbp]
  \includegraphics[width=0.8\textwidth]{Table_2.png} 
  \centering
  \caption{Data Table for Reverse Saturation Current} 
  \label{tab:tab2} 
\end{figure}


\begin{figure}[htbp]
  \includegraphics[width=0.8\textwidth]{Table_2.png} 
  \centering
  \caption{Data Table for Weak Forward Bias} 
  \label{tab:tab3} 
\end{figure}

\begin{figure}[htbp]
  \includegraphics[width=0.8\textwidth]{Table_4.png} 
  \centering
  \caption{Data Table for Strong Forward Bias} 
  \label{tab:tab4} 
\end{figure}





\end{document}