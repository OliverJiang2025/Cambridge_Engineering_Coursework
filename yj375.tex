\documentclass[12pt]{article}

\topmargin 0.0cm 
\oddsidemargin 0.0in 
\evensidemargin0.0in
\textheight 22cm 
\textwidth  17cm 
\headheight 0in 
\headsep 0in
\parindent0in

\usepackage{amsmath}
\usepackage{amsfonts}
\usepackage{amssymb}
\usepackage{graphicx}
\graphicspath{{./Figures/}}
\usepackage{tikz}
\usepackage{amssymb}
\usepackage{circuitikz}
\usepackage{setspace}
\usepackage{subcaption}
\doublespacing
\begin{document}

%\hspace{1cm}

\begin{center}
\Large{\bf CAMBRIDGE UNIVERSITY ENGINEERING DEPARTMENT}
\end{center}

\vskip 1cm

\begin{center}
\large{\bf Part IIA Laboratory Report}
\end{center}

\vskip 1cm
\begin{center}
\fbox{\rule[0.0cm]{0cm}{1.0cm}
 \large{\bf 3B3 DC-DC Converters}\rule[-0.75cm]{0cm}{1.0cm} }
\end{center}

\vskip 1cm

\begin{center}

Name: Yongqing Jiang \\
\vskip 0.2cm
CRSid: yj375  \\
\vskip 0.2cm
 College: Peterhouse \\
\vskip 0.2cm

Date of Experiment: Nov 2025

\end{center}

\vskip 1cm

\newpage
\section*{Exercise 1: Diode Bridge AC-DC Rectifier}
The scope readings for exercise 1 are shown by
Figure \ref{fig:ex1}.
\begin{figure}[htbp] 
    \centering 
    \begin{subfigure}[t]{0.3\textwidth}
        \centering
        \includegraphics[width=\linewidth]{Figure1-2.jpg} 
        \caption{Reading of 1-2}
        \label{fig:1-2}
    \end{subfigure}
    \hspace{2cm} 
    \begin{subfigure}[t]{0.3\textwidth}
        \centering
        \includegraphics[width=\linewidth]{Figure1-3.jpg}
        \caption{Reading of 1-3}
        \label{subfig:1_3}
    \end{subfigure}
    \caption{Scope readings of Exercise 1}
    \label{fig:ex1} 
\end{figure}

\section*{Exercise 2: DC-DC Buck Converter}
The testing schematic of the DC-DC Buck Converter
is shown by Figure \ref{fig:buck}.

\begin{figure}[htbp]
    \centering
    \includegraphics[width = 1\linewidth]{Buck.jpg}
    \caption{Schematic of DC-DC Buck Converter}
    \label{fig:buck}
\end{figure}

The measurements are carried by four channels of
the oscilliscope.
\begin{enumerate}
    \item CH1: Gate pulse voltage $v_G$
    \item CH2: Diode voltage $v_D$
    \item CH3: Output DC voltage $V_o$
    \item CH4: Inductor current $i_L$
\end{enumerate}
Note that CH4 measures the current via a current probe and
shows the reading in V on display.

\begin{figure}[htbp] 
    \centering 
    \begin{subfigure}[t]{0.4\textwidth} % 子图宽度可适度扩大
        \centering
        \includegraphics[width=\linewidth]{Figure2-2-1.jpg} 
        \caption{Reading with $V_o$ = 12V}
        \label{fig:2-2-1}
    \end{subfigure}
    \hspace{2cm}
    \begin{subfigure}[t]{0.4\textwidth}
        \centering
        \includegraphics[width=\linewidth]{Figure2-2-2.jpg}
        \caption{Reading with $V_o$ = 18V}
        \label{fig:2-2-2}
    \end{subfigure}
    \caption{Scope readings of section 2-2}
    \label{fig:2-2}
\end{figure}

In Section 2-2, scope readings are taken with $V_o$ = 12V
and $V_o$ = 18V. Duty ratio D = 0.45 for $V_o$ = 12V, 
D = 0.68 for $V_o$ = 18V. 

According to theory, the voltage 
transfer ratio of a buck converter $M(D) = \frac{V_o}{V_d} = D$.
So, the theoretical values of the two measurements should be
$\frac{12}{30} = 0.4$ and $\frac{18}{20} = 0.6$ respectively.

We can see that the measured duty ratios are about 10\% larger
than the theoretical values for both cases. 
This is due to non-ideal parts
of circuits, mainly the extra voltage drop due to the diode 
and the parasitic resistance of the inductor.

Thus, for both ON and OFF states of the transistor,
the DC output = $V_o - \Delta V$ is kept 
constant actually. $V_o = DV_d$ should be larger than
theoretical value of compensate this effect. As result,
duty ratio is measured to be larger.

In the next session, load resistor and
inductors are explored by switching more inductors on. 
The reading of scope
is shown in Figure \ref{fig:2-3}.


\begin{figure}[htbp] 
    \centering 
    \begin{subfigure}[t]{0.3\textwidth} % 子图宽度可适度扩大
        \centering
        \includegraphics[width=\linewidth]{Figure2-3-1.jpg} 
        \caption{Reading with $L_3$ ON}
        \label{fig:2-3-1}
    \end{subfigure}
    \hfill
    \begin{subfigure}[t]{0.3\textwidth}
        \centering
        \includegraphics[width=\linewidth]{Figure2-3-2.jpg}
        \caption{Reading with $L_3$ and $L_2$ ON}
        \label{subfig:2-3-2}
    \end{subfigure}
    \hfill
    \begin{subfigure}[t]{0.3\textwidth}
        \centering
        \includegraphics[width=\linewidth]{Figure2-3-3.jpg}
        \caption{Reading with $L_3$, $L_2$, and $L_1$ ON}
        \label{subfig:2-3-3}
    \end{subfigure}
    \caption{Scope readings of section 2-3}
    \label{fig:2-3}
\end{figure}
To find the load resistance $R_L = \frac{V_o}{I_o}$, output
current $I_o$ should be obtained from the readings of 
CH4, which measures the inductor current $i_L$. By making
the assumption that the output smoothing capacitor is large
enough that all ripple current goes into the capacitor, 
$I_o = i_{L\text{avg}}$. This average value can be obtained
by the built-in measurements of the oscilliscope.
From Figure \ref{fig:2-2-1} and \ref{fig:2-2-2}, we have 
$I_o = 1.9285A$ for $V_o = 12V$ and $I_o = 5.3808A$ for $V_o = 18V$.
Thus, the load resistances are calculated as $6.22 \Omega$ and $3.34 \Omega$
for this two cases respectively. There is a huge discrepancy as expected. 
Only one measurement is taken per case and the parasitic resistance is 
not accounted for. We could improve this estimate of load resistance
by taking multiple measurements and averaging them, or by considering 
the parasitic resistance of the inductor and diode voltage drop.

For now, we can only conclude that the load resistance is somewhere
between $3 \Omega$ and $6 \Omega$.

After that, interest focuses on the inductance $L_1$, $L_2$, and $L_3$.
This can be found by considering the
peak-to-peak ripple current $\Delta I_L$.

From theory in Continuous Conduction Mode, $\Delta I_L = \frac{V_d D(1-D)}{f_s L}$, where
$f_s$ is the switching frequency, 100kHz. Rearranging gives
$L = \frac{V_d D(1-D)}{f_s \Delta I_L}$.



\newpage
\section*{Exercise 3: DC-DC Boost Converter}


\end{document}